% -*- coding: utf-8; -*-
\chapter{Proposal}
\label{cha:Proposal}

Equation example 1:

\begin{equation}
\begin{split}
\min_u \int_{x_i\in X}\int_{x_j\in X} q_{ij} u_i u_j da da + \int_{x_i\in X}||x' - x_i|| u_i da \\
s.t. \ \ \ u\in[0,1] \ \ \land  \ \ \int_{x_i\in X}u da = a_0,
\end{split}
\end{equation}

Equation exmaple 2:

\begin{equation}
\begin{split}
\min_{\mathbf{u}} \alpha \mathbf{u}^T \mathbf{A}^T \mathbf{Q} \mathbf{A} \mathbf{u} +  \beta \mathbf{d}^T a' \mathbf{A} \mathbf{u} + \gamma \mathbf{u}^T \mathbf{G}^T \mathbf{G} \mathbf{u} + \delta\mathbf{f}^T a' \mathbf{A} \mathbf{u} \\
s.t. \ \ \ \mathbf{0} \leq \mathbf{u} \leq \mathbf{1} \land \mathbf{a}^T\mathbf{u}=a_0.
\end{split}
\end{equation}

Equation example 3:
\begin{align}
\mathbf{G}=(g_{ij}) = \left\lbrace
\begin{array}{ll}
\sum_{f_k\in N_f(f_i)} l_{ik} & i=j\\
-l_{ij} & e_{ij}\in E\\
0 & \text{otherwise}
\end{array}
\right.
\end{align}

\lstinputlisting[label=mean,title={Mean Filter},caption={Mean Filter},language=R]{codes/mean.R}

%% Poruguese algorithm
%\begin{algorithm}
%\DontPrintSemicolon
%\Entrada{Malha e quantidade de pontos a ser amostrado}
%\Saida{Pontos amostrados na malha}
%\BlankLine
%\emph{Crie um vetor de números randômicos entre $[0,1]$ com a %quantidade de pontos a ser amostrada e ordene-o}\;
%\emph{Calcule a área total dos triângulos da malha}\;
%\For{$i=0$ \KwTo numeroDePontos} {
%  \emph{Navegue entre as faces acumulando a sua $\frac{area}{areaTotal}$ até achar a face com valor acumulado $\geqslant$ numerosRandomicos[i]}\;
%  \emph{Pegue um ponto randômico dentro da face utilizando o %método de Turk e adicione no vetor do resultado}\;
%}
%\caption{Escolha das amostras inicias}\label{alg:sampling}
%\end{algorithm}\DecMargin{1em}

%% enlgish algorithm
\begin{algorithm}
\DontPrintSemicolon
\KwIn{Malha e quantidade de pontos a ser amostrado}
\KwOut{Pontos amostrados na malha}
\BlankLine
\emph{Crie um vetor de números randômicos entre $[0,1]$ com a quantidade de pontos a ser amostrada e ordene-o}\;
\emph{Calcule a área total dos triângulos da malha}\;
\For{$i=0$ \KwTo numeroDePontos} {
  \emph{Navegue entre as faces acumulando a sua $\frac{area}{areaTotal}$ até achar a face com valor acumulado $\geqslant$ numerosRandomicos[i]}\;
  \emph{Pegue um ponto randômico dentro da face utilizando o método de Turk e adicione no vetor do resultado}\;
}
\caption{Escolha das amostras inicias}\label{alg:sampling}
\end{algorithm}\DecMargin{1em}






